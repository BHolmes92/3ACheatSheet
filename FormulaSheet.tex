%%%%%%%%%%%%%%%%%%%%%%%%%%%%%%%%%%%%%%%%%%%%%%%%%%%%%%%%%%%%%%%%%%%%%%
% writeLaTeX Example: A quick guide to LaTeX
%
% Source: Dave Richeson (divisbyzero.com), Dickinson College
% 
% A one-size-fits-all LaTeX cheat sheet. Kept to two pages, so it 
% can be printed (double-sided) on one piece of paper
% 
% Feel free to distribute this example, but please keep the referral
% to divisbyzero.com
% 
%%%%%%%%%%%%%%%%%%%%%%%%%%%%%%%%%%%%%%%%%%%%%%%%%%%%%%%%%%%%%%%%%%%%%%
% How to use writeLaTeX: 
%
% You edit the source code here on the left, and the preview on the
% right shows you the result within a few seconds.
%
% Bookmark this page and share the URL with your co-authors. They can
% edit at the same time!
%
% You can upload figures, bibliographies, custom classes and
% styles using the files menu.
%
% If you're new to LaTeX, the wikibook is a great place to start:
% http://en.wikibooks.org/wiki/LaTeX
%
%%%%%%%%%%%%%%%%%%%%%%%%%%%%%%%%%%%%%%%%%%%%%%%%%%%%%%%%%%%%%%%%%%%%%%

\documentclass[a4paper]{article}
\usepackage{amssymb,amsmath,amsthm,amsfonts}
\usepackage{multicol,multirow}
\usepackage{calc}
\usepackage{ifthen}
\usepackage[landscape]{geometry}
\usepackage[colorlinks=true,citecolor=blue,linkcolor=blue]{hyperref}
\usepackage{graphicx}


\ifthenelse{\lengthtest { \paperwidth = 11in}}
    { \geometry{top=.5in,left=.5in,right=.5in,bottom=.5in} }
	{\ifthenelse{ \lengthtest{ \paperwidth = 297mm}}
		{\geometry{top=1cm,left=1cm,right=1cm,bottom=1cm} }
		{\geometry{top=1cm,left=1cm,right=1cm,bottom=1cm} }
	}

\pagestyle{empty}
\makeatletter
\renewcommand{\section}{\@startsection{section}{1}{0mm}%
                                {-1ex plus -.5ex minus -.2ex}%
                                {0.5ex plus .2ex}%x
                                {\normalfont\large\bfseries}}
\renewcommand{\subsection}{\@startsection{subsection}{2}{0mm}%
                                {-1explus -.5ex minus -.2ex}%
                                {0.5ex plus .2ex}%
                                {\normalfont\normalsize\bfseries}}
\renewcommand{\subsubsection}{\@startsection{subsubsection}{3}{0mm}%
                                {-1ex plus -.5ex minus -.2ex}%
                                {1ex plus .2ex}%
                                {\normalfont\small\bfseries}}
\makeatother
\setcounter{secnumdepth}{0}
\setlength{\parindent}{0pt}
\setlength{\parskip}{0pt plus 0.5ex}
% -----------------------------------------------------------------------

\title{MTE321 Formulas}

\begin{document}

\raggedright
\footnotesize

\begin{center}
     \Large{\textbf{MTE321 Formulas}} \\
\end{center}
\begin{multicols}{3}
\setlength{\premulticols}{1pt}
\setlength{\postmulticols}{1pt}
\setlength{\multicolsep}{1pt}
\setlength{\columnsep}{2pt}


\section{Stresses}
\subsection{Deformation Elongation}
\begin{align*}
    \delta = \frac{FL}{EA}\\
    \delta = \frac{\sigma L}{E}
\end{align*}


\subsection{Torsional Formulas}
\subsubsection{Stress}
R is the radial distance
\begin{align*}
    \tau = \frac{Tr}{J}\\
    Z_p = \frac{J}{c}\\
    \tau_{max} = \frac{T}{Z_p}\\
    \textit{Hollow: }J = \frac{\pi}{2}(C^4 - C_i^4)\\
    \textit{Solid: }J = \frac{\pi}{2}C^4\\    
\end{align*}

\subsubsection{Deformation}
$\theta$ is the angle of twist across L\\
For non-circular shafts K is section polar second moment of area and Q the section polar modulus
\begin{align*}
	T = \frac{P_W}{\omega} \hspace{4mm} T_{lb\cdot in}=63000\frac{P_{hp}}{\omega}\\
    \theta = \frac{TL}{GJ}\\
    \textit{Non-Circular }\tau = \frac{T}{Q}\\
    \textit{Non-Circular }\theta = \frac{TL}{GK}
\end{align*}
\subsubsection{Thin-Walled Closed Tubes}
A = median area boundary, U is length of median boundary
\begin{align*}
	K = \frac{4A^2t}{U}\\
	Q = 2tA\\
\end{align*}
\subsection{Shear Stress}
V section shear force, Q is the first moment area, and t is the section thickness
\begin{align*}
	\tau_{(y)} = \frac{VQ}{It}\\
	\textit{Rectangular Beam }\tau_{max} = \frac{3V}{2A}\\
	\textit{Solid Round Beam }\tau_{max} = \frac{4V}{3A}\\
	\textit{Hollow Round Beam }\tau_{max} = \frac{2V}{A}\\
	Q = A_p\bar{y}\\
	\bar{y} = \textit{Distance to central axis}\\
	A_p = \frac{1}{12}t\cdot h\textit{ rectangle}\\
\end{align*}

\subsubsection{Beam Bending}
M is the moment at the section, y is the distance from the neutral axis
\begin{align*}
    \sigma_{y} = -\frac{My}{I}\\
    \sigma_{max} = \frac{M}{S}\\
\end{align*}

\section{Stress Concentrations}

\subsection{Stress Concentration Factor}
K\textsubscript{t} is material and loading dependent, values greater than 3 are a waste
\begin{align*}
    \sigma_{max} = K_t\sigma_{nom}
\end{align*}
\subsubsection{Curved Beam Bending}
R = $\frac{A}{ASF}$\\
r = distance to required stress location\\
r\textsubscript{c}= centroid distance\\
A = cross-sectional area\\
\begin{align*}
	\sigma_{(r)} = \frac{M(R-r)}{Ar(r-R)}
\end{align*}
\subsection{Thermal Strain}
\begin{align*}
	\textit{Fixed between two walls }\epsilon_x^m=-\alpha\Delta T\\
	\epsilon_x^t = \alpha\Delta T\\
\end{align*}
\subsection{Principle Stresses}
\begin{align*}
	tan2\theta_\sigma = \frac{2\tau_{xy}}{\sigma_x-\sigma_y}\\
	\sigma_{1,2} = \frac{\sigma_x + \sigma_y}{2} \pm \sqrt{\left(\frac{\sigma_x-\sigma_y}{2}\right)^2 + \tau^2_{xy}}\\
	\textit{Max }\sigma_{norm} = \frac{1}{2}(\sigma_x + \sigma_y) + \sqrt{\left[\frac{1}{2}\sigma_x-\sigma_y\right]^2 + \tau_{xy}^2}\\
	\textit{Min }\sigma_{norm} = \frac{1}{2}(\sigma_x - \sigma_y) - \sqrt{\left[\frac{1}{2}\sigma_x-\sigma_y\right]^2 + \tau_{xy}^2}\\
	\tau_{max} = \pm\sqrt{\left(\frac{\sigma_x-\sigma_y}{2}\right)^2 + \tau_{xy}^2}\\
\end{align*}
\section{Static Loads}
\subsection{Effective Stress}
\begin{align*}
\textit{Tresca: }\sigma^` = \frac{\sigma_1-\sigma_3}{2}\\
\textit{Von Mises: }\sigma_e = \frac{1}{\sqrt{2}}\sqrt{(\sigma_1 - \sigma_2)^2 + (\sigma_1 - \sigma_3)^2 + (\sigma_2 - \sigma_3)^2}
\end{align*}
\subsection{Static Loading}
\begin{align*}
N = \frac{s_y}{2\tau_{max}}\\
N = \frac{s_y}{\sigma_e}\\
\textit{If brittle}\\
N = \frac{S_{ut}}{K_t\sigma_1}\\
N = \frac{S_{uc}}{K_t\sigma_3}\\
\sigma_1 > \sigma_2 > \sigma_3\\
\end{align*}
\section{Design Factors}
\subsection{A\textsubscript{95} Equivalence}
\begin{align*}
\textit{Equivalent Diameter: }D_e=0.370D\\
\textit{General: }0.0766D_e^2\\
\end{align*}
\section{Design: Dynamic Loads}
\subsection{Loading}
\begin{align*}
	\sigma_m =\textit{mean stress  } = \frac{\sigma_{max} + \sigma_{min}}{2} \\
	\sigma_a = \textit{stress amplitude  } = \frac{\sigma_{max} - \sigma_{min}}{2}\\
	R = \textit{stress ratio} = \frac{\sigma_{min}}{\sigma_{max}}\\
	A = \textit{stress ratio  } = \frac{\sigma_a}{\sigma_m}\\
	\textit{Loading Cycle: preriod between peaks }
\end{align*}
\subsection{Stress}
\subsubsection{Periodic}
Fluctuating $\sigma_m \neq 0$ , R = -1\\
Pulsating $\sigma_{min} = 0$, R =1\\
\subsubsection{Endurance Limit}
s\textsubscript{a} =Stress Amplitude Level\\
N: number of cycles to failure\\
s\textsubscript{n} = fatigue limit\\
Assume s\textsubscript{n} = 0.5s\textsubscript{u} if no data\\
\begin{align*}
s_a = s_nN^b\\
s^`_n = C_mC_{st}C_RC_Ss_n\\
\textit{C\textsubscript{S} only in bending}\\
\end{align*}
s\textsubscript{n} from table appendix 3\\
C\textsubscript{m} material flaws\\
C\textsubscript{R} Reliability Factor Assume 0.99 reliability\\
C\textsubscript{s} = size factor (5-12,5-4 circular),(5-13 for other)\\
\includegraphics[scale=0.5]{ActualEnduranceLimit.png}
\subsection{Goodman Method}
\subsubsection{Dynamic Loads Compressive}
\begin{align*}
(\sigma_m \leq 0)\\
\textit{Von Mises: } N_1 = \frac{s^`_n}{K_t\sigma^`_a}\\
\textit{Tresca: } N_1 = \frac{s^`_n}{K_t\sigma^`_a}\\
\end{align*}
\subsubsection{Dynamic Loads Tensile}
\begin{align*}
(\sigma_m > 0)\\
\textit{Von Mises: } \frac{K_t\sigma^`_a}{s^`n} + \frac{\sigma^`_m}{s_u} = \frac{1}{N_1}\\
\textit{Tresca: } \frac{2K_t}{s^`_n}(\tau_a)_{max} + \frac{4}{3s_u}(\tau_m)_{max} = \frac{1}{N_1}\\
\end{align*}
\subsubsection{Dynamic Yield Test}
\begin{align*}
\textit{for low }\sigma_a\textit{ high }\sigma_m\\
\textit{Von Mises: } \frac{K_t\sigma^`_a}{s_{y}} + \frac{K_t\sigma^`_m}{s_y} = \frac{1}{N_2}\\
\textit{Tresca: } \frac{2K_t}{s_{sy}}(\tau_a)_{max} + \frac{2K_t}{S_{sy}}(\tau_m)_{max} = \frac{1}{N_2}
\end{align*}
Effective safety factor is $<$ of N\textsubscript{1} and N\textsubscript{2}
\section{Gears}
Table 8-1\\
\subsection{Pitch Line Speed}
\begin{align*}
V_T = \frac{\pi D\cdot n_p}{12}
\end{align*}
\section{Gears}
\subsection{Spur Gears}
\begin{align*}
\textit{Center Distance }C=R_P + R_G\\
\textit{Speed of Gears: } \frac{n_p}{n_G}=\frac{N_G}{N_P}\\
\textit{Common Speed: } v_T = R_1\omega_1 = R_2\omega_2 \\
\textit{Tangental Acceleration: }a_T = R_1\alpha_1 = R_2\alpha_2\\
\textit{Velocity Ratio: } VR = \frac{R_G}{R_P} \geq 1 = \frac{N_G}{N_P}=\frac{n_p}{n_G}=\frac{\omega_P}{\omega_G}\\
\textit{Circular Pitch: } p = \frac{\pi D}{N}\\
\textit{Contact Ratio: }m_f = \frac{\sqrt{R_{oP}^2 - R_{{bP}}}+\sqrt{R_{oG}^2 - R_{{bG}}}-c\sin\phi}{p\cos\phi}\\
P=T\omega\\
\textit{backlash:}=w-t\\
w\rightarrow\textit{Tooth Space, distance pitch circle travels between teeth}\\
\end{align*}
\subsection{Helical Gears}
\begin{align*}
\textit{Circular/Transverse Pitch: }p = \frac{\pi}{P_d}\\
\textit{Normal Circular: }p_n = p\cos\psi \\
\textit{Axial Pitch: }p_x = \frac{p_t}{\tan\psi}\\
\textit{Pitch Diameter: }D_G = \frac{N}{P_d}\\
\textit{Normal Pressure Angle: }\phi_n = \tan^{-1}(\tan\phi_t \cdot \cos\psi)\\
\textit{Diametral Pitch: }P_{d} = \frac{N}{D}\\
\textit{Axial Pitches in Face: } \frac{F_w}{P_x}\\
\textit{Normal Diametral Pitch: } P_{nd} = \frac{P_d}{\cos\psi}
\end{align*}
\subsection{Gear Train}
\begin{align*}
TV_{nom} = \frac{n_{in}}{n_{out}}\\
\end{align*}
\subsection{Racks}
\begin{align*}
\textit{Velocity of Rack: }V_R = V_T = R_p\omega_p = \left(\frac{D_p}{2}\right)\omega_p\\
\textit{Displacement of Rack: }s = \frac{D_p}{2}\theta_p\\
\end{align*}
\section{Gear Stress}
\subsection{Bending Stress}
\begin{align*}
\sigma_t = \frac{W_tP_d}{FJ}\\
J = \frac{Y}{K_t}\\
\textit{Y = Lewis Form Factor}\\
s_t = \sigma_tK_0K_SK_mK_BK_v\\
\includegraphics[scale=0.45]{bendingStressFactors.png}
\end{align*}
\subsection{Contact Stress}
\begin{align*}
\sigma_c = C_P\sqrt{\frac{W_t}{FD_PI}}\\
s_c = C_p\sqrt{\frac{W_tK_oK_SK_mK_v}{FD_PI}}\\
\includegraphics[scale=0.45]{contactStressFactors.png}
\end{align*}
\section{Power Capacity}
\begin{align*}
\textit{Bending}\\
P_{CAP} = \frac{s_{at}Y_NFJn_pD_P}{(126000)P_d(SF)K_RK_OK_SK_mK_BK_v}\\
\textit{Contact}\\
P_{CAP} = \frac{n_pFI}{126000K_oK_sK_mK_v}\left[\frac{s_{ac}D_PZ_N}{(SF)K_RC_P}\right]^2\\
\end{align*}
\section{Driven Forces}
\subsection{Spur Gears}
\begin{align*}
\textit{Tangental Forces: } W_t = \frac{T}{\frac{D}{2}}\\
\textit{Radial Force: } W_r = W_t\tan\phi\\
\end{align*}
\subsection{Chains/Belts}
\begin{align*}
\textit{Chain/Sproket: }\\
F_{cx} = F_c\cos\phi\hspace{4mm}F_{cy}=F_c\sin\phi\\
\textit{Belt Sheave}\\
\textit{V-Belt: }F_B=1.5\frac{T}{\frac{D}{2}}\\
\textit{Flat Belt: }F_B=2\frac{T}{\frac{D}{2}}\\
VR = \frac{\omega_{driving}}{\omega_{driven}}\\
\end{align*}
\subsection{Shoulder Fillets}
\begin{align*}
\textit{Sharp Fillet (Bearing): }K_t = 2.5\\
\textit{Rounded Fillet: }K_t = 1.5\\
\textit{Retaing Ring: }K_t = 3.0\\
\textit{Profile Keyseat: }K_t = 2.0\\
\textit{Sled Runner Keyseat: }K_t = 1.6\\
\end{align*}
\section{Design}
\subsection{Shear Stress}
\begin{align*}
\tau_d = K_t\left(\frac{4V}{3A}\right) \hspace{4mm} N = 0.577\frac{s`_n}{\tau_d} \hspace{4mm} A = \frac{\pi D^2}{A}\\
D = \sqrt{\frac{2.94K_tVN}{s`_n}}\\
\end{align*}
\subsection{Bending\textbackslash Torsion}
\begin{align*}
M = \sqrt{M_x^2 + M_y^2}\\
\left(\frac{\sigma}{s`_n}\right)^2 + \left(\frac{\tau}{s_{ys}}\right)^2 = 1\\
s_{ys} = \frac{s_y}{\sqrt{3}} \hspace{4mm} s_{ys} = \frac{s_y}{\sqrt{3}}\\
\left(\frac{K_TN\sigma}{s`_n}\right)^2+\left(\frac{N_T}{s_{ys}}\right)^2 =1\\
D^3 = \frac{32N}{\pi}\sqrt{\left(\frac{K_tM}{s`_n}\right)^2 + \frac{3}{4}\left(\frac{\tau}{s_y}\right)^2}\\
\textit{Steel Shafts: }s`_n = s_n C_R C_s\\
\end{align*}
\end{multicols}
\end{document}
